\vsssub
\subsubsection{Point output post-processor for GrADS} \label{sec:gxoutp}
\vsssub

\proddefH{gx\_outp}{gxoutp}{gx\_outp.ftn}
\proddeff{Input}{gx\_outp.inp}{Traditional configuration file.}{10} (App.~\ref{sec:config181})
\proddefa{mod\_def.ww3}{Model definition file.}{20}
\proddefa{out\_pnt.ww3}{Raw point output data.}{20}
\proddeff{Output}{standard out}{Formatted output of program.}{6}
\proddefa{ww3.spec.grads}{GrADS data file with spectra and source terms.}{30}
\proddefa{ww3.mean.grads}{File with mean wave parameters.}{31}
\proddefa{ww3.spec.ctl}{GrADS control file.}{32}

\vspace{\baselineskip} 
\noindent
This post-processor is intended to generate data files with which GrADS (see
previous section) can plot polar plots of spectra and source terms. To achieve
this, spectra and source terms are store as "longitude-latitude" grids. For
each output point a different name is generated for the data, typically {\F
loc{\it nnn}}. When the data file is loaded in GrADS, the variable {\F loc001}
will contain a spectral grid for the first requested output point at level 1,
the input source term at level 2, etc. For the second output point the data is
stored in {\F loc002} etc. The actual output point names are passed to GrADS
through the control file {\file ww3.spec.ctl}. Wave heights and environmental
data are obtained from {\file ww3.mean.grads} The user, however, need not be
aware of the details of the GrADS data files and data storage. The GrADS
scripts {\file spec.gs}, {\file source.gs} and {\file 1source.gs} are provided
to automatically generate spectral plots from the output files of this
post-processor.

Note: for the GrADS scripts to work properly, the names of the output points
should not contain spaces.

\pb
