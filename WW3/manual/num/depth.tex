\vssub
\subsection{~Depth variations in time} \label{sub:num_depth}
\vssub

Temporal depth variations result in a change of the local wavenumber
grid. Because the wavenumber spectrum is invariant with respect to temporal
changes of the depth, this corresponds to a simple interpolation of the
spectrum from the old grid to the new grid, without changes in the spectral
shape. As discussed above, the new grid simply follows from the globally
invariant frequency grid, the new water depth $d$ and the dispersion relation
Eq. (\ref{eq:disp}). The time step of updating the water level is generally
dictated by physical time scales of water level variations, but not by
numerical considerations \citep{tol:GAOS98b}.

The interpolation to the new wavenumber grid is performed with a simple
conservative interpolation method. In this interpolation the old spectrum is
first converted to discrete action densities by multiplication with the
spectral bin widths. This discrete action then is redistributed over the new
grid cf.\ a regular linear interpolation. The new discrete actions then are
converted into a spectrum by division by the (new) spectral bin widths. The
conversion requires a parametric extension of the original spectrum at high
and low frequencies because the old grid generally will not completely cover
the new grid. Energy/action in the old spectrum at low wavenumbers that are
not resolved by the new grid is simply removed. At low wavenumbers in the new
grid that are not resolved by the old grid zero energy/action is assumed. At
high wavenumbers in the new grid the usual parametric tail is applied if
necessary. The latter correction is rare, as the highest wavenumbers usually
correspond to deep water.

In practical applications the grid modification is usually relevant for a
small fraction of the grid points only. To avoid unnecessary calculations, the
grid is transformed only if the smallest relative depth $kd$ in the discrete
spectrum is smaller than 4. Furthermore, the spectrum is interpolated only if
the spatial grid point is not covered by ice, and if the largest change of
wavenumber is at least $0.05 \Delta k$.
